 % use the "wcp" class option for workshop and conference
 % proceedings
 %\documentclass[gray]{jmlr} % test grayscale version
 %\documentclass[tablecaption=bottom]{jmlr}% journal article
 \documentclass[tablecaption=bottom,wcp]{jmlr} % W&CP article

 % The following packages will be automatically loaded:
 % amsmath, amssymb, natbib, graphicx, url, algorithm2e

 %\usepackage{rotating}% for sideways figures and tables
 %\usepackage{longtable}% for long tables

 % The booktabs package is used by this sample document
 % (it provides \toprule, \midrule and \bottomrule).
 % Remove the next line if you don't require it.
\usepackage{booktabs}
 % The siunitx package is used by this sample document
 % to align numbers in a column by their decimal point.
 % Remove the next line if you don't require it.
\usepackage[load-configurations=version-1]{siunitx} % newer version
 %\usepackage{siunitx}

 % The following command is just for this sample document:
\newcommand{\cs}[1]{\texttt{\char`\\#1}}% remove this in your real article

 % Define an unnumbered theorem just for this sample document for
 % illustrative purposes:
\theorembodyfont{\upshape}
\theoremheaderfont{\scshape}
\theorempostheader{:}
\theoremsep{\newline}
\newtheorem*{note}{Note}

 % change the arguments, as appropriate, in the following:
\jmlrvolume{1}
\jmlryear{20XX}
\jmlrsubmitted{submission date}
\jmlrpublished{publication date}
\jmlrworkshop{workshop title} % W&CP title

 % The optional argument of \title is used in the header
\title[SUNNY-OASC]{SUNNY with Algorithm Configuration}


 % Two authors with the same address
  \author{\Name{Tong Liu} \Email{t.liu@unibo.it}\\
   \Name{Roberto Amadini} \Email{roberto.amadini@unimelb.edu.au}\\
   \Name{Jacopo Mauro} \Email{mauro.jacopo@gmail.com}\\
    }



 % Authors with different addresses:
 % \author{\Name{Author Name1} \Email{abc@sample.com}\\
 % \addr Address 1
 % \AND
 % \Name{Author Name2} \Email{xyz@sample.com}\\
 % \addr Address 2
 %}


 %\editors{Editor One and Editor Two}% for multiple editors

\begin{document}

\maketitle



\section{Description}
This proposed solution is an improvement of SUNNY-AS \cite{DBLP:conf/cilc/AmadiniBGLM15,sunnyas} with the ideas suggested by the Works \cite{DBLP:conf/lion/LindauerBH16,Kohavi97wrappersfor}.

SUNNY-AS is an per instance algorithm scheduling strategy based on K-NN techniques. The Work \cite{DBLP:conf/lion/LindauerBH16} has demonstrated that how a training step, by studying the value $K$ and the number of solvers, can improve SUNNY's performance significantly. In this Work, we have proposed two solutions, `autok' and `fkvar'. The `autok' re-implemented TSUNNY mentioned in \cite{DBLP:conf/lion/LindauerBH16} by considering only the value $K$ (ignoring the number of solvers). The `fkvar' instead trains for both value $K$ and optimal features by using a wrapper method \cite{Kohavi97wrappersfor}, which selects the best combination of $K$ and features, such that, SUNNY enhances the most on training data. 

Our previous experiments \cite{DBLP:conf/cilc/AmadiniBGLM15} suggested that a handful subset of features (eg: 5) is often enough for SUNNY to obtain a competitive performance, as such, in `fkvar' we fixed such amount of feature to select. In order to guarantee an acceptable execution runtime, for the `fkvar' approach, we have taken up to $1500$ representative instances from training set as effective instances, and we also fixed the interval of $K$ as [3,30]. In the end of execution, we re-run `autok' $K \in [3,80]$ for a backup, i.e. if SUNNY runs better with entire features, we then adopt the combination of $K$ and the whole feature set instead. Differently, in the `autok' version, we consider the full training set as effective training data.

\section{Setup Instruction}

The source code is available at \cite{sunnyoasc} which requires Python v2.x. There are five folders, `data' and `results' contain oasc-challenge data and solution results respectively. `src' contains the original SUNNY-AS scripts from \cite{sunnyas}, `oasc' contains scripts who coordinate those in `src' for training and testing. In the end, in the folder `main', there have been placed the scripts that automatically call scripts in `oasc' for different scenarios. 

The program runs training and testing in sequence, let us take `autok' approach as execution example. In the folder `main', launch the command ``sh make\_oasc\_tasks.sh $<$ tasks.txt'' to create tasks. Then train scenarios with ``sh oasc\_train.sh run\_autok tasks.txt''. After training, run testing with command ``sh make\_oasc\_tasks.sh $<$ tasks.txt'' then, ``sh oasc\_test.sh autok tasks.txt''. Whereas, to run fkvar approach, it is sufficient to replace literally `autok' by `fkvar' in the previous commands. 

\bibliography{reference}

\appendix

\end{document}
